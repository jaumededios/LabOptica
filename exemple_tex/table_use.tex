
\documentclass{article}
\usepackage[utf8]{inputenc}
\usepackage{csvsimple}
\usepackage{makecell}
% MWE del csvsimple per a fer una taula 

% Per a la taula donada, les opcions són
% \lambda, \delta, \interpguio, \interpminsq, \n
% totes autoexplicatives


% Canviant el nom de la taula per
% ../4/OUT/TBL/cd.csv
% carreguem la del cadmi


%
% Detall, t'he donat totes les dades en una unic arxiu,
% però no cal que es posin totes en una taula, en podries
% crear una amb els indexs de refraccio, una amb les
% interpolacions... com tu creguis!

\begin{document}

\begin{table}[!h]
\renewcommand{\arraystretch}{1.3}
\centering
\begin{tabular}{c c c c}
    \hline 


    Longitud d'ona [$\mathring A$] & 
    Índex de refracció &
    Desviació mínima [º] &
    \thead{Interpolació de $\lambda$ per\\
            mínims quadrats [$\mathring A$]}


    \\
    \hline

    %aquí canvies els \loquesigui per a que es mostrin
    % i el nom de l'arxiu
    \csvreader[head to column names]{../4/OUT/TBL/hg.csv}{}
      {\lambda & \n & \delta & \interpminsq \\}


    & \\[-3ex]

    \hline

\end{tabular}
\caption{Taula que demostra que la Irina és un panda}
\label{tab:panda}
\end{table}

\end{document}